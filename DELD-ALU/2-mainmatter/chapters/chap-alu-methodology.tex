\chapter{Methodology}

The project followed a \textbf{four-stage structured design methodology}, ensuring that all functional and timing requirements were met before proceeding to physical construction.

\section{Problem Analysis}

The primary objective is to design and implement a comprehensive 4-bit Arithmetic Logic Unit (ALU) using discrete digital components (basic logic gates and flip-flops) to execute essential arithmetic and bitwise logical operations.

Key considerations during the problem analysis phase included:
\begin{itemize}
    \item Identifying the required arithmetic operations (addition, subtraction, multiplication, division)
    \item Defining the logical operations (AND, OR, XOR)
    \item Determining the control signal requirements for operation selection
    \item Establishing the modular architecture for the ALU design
    \item Analyzing timing and propagation delays
    \item Planning for testability and debugging
\end{itemize}

\section{Circuit Design}

The circuit design phase involved creating detailed schematics for each functional block of the ALU. The design was modular, consisting of:

\begin{table}
\captionabove{ALU Functional Blocks}
\centering
\begin{tabular}{ll}
\toprule
\textbf{Module} & \textbf{Description} \\
\midrule
Arithmetic Unit (AU) & Addition, subtraction, multiplication, division \\
Logic Unit (LU) & Bitwise AND, OR, XOR operations \\
Control Unit & Operation selector using multiplexers \\
Output Stage & Result propagation and flag generation \\
\bottomrule
\end{tabular}
\label{tab:alu-blocks}
\end{table}

\section{Simulation}

\textbf{Tools used:} Logisim, Proteus Design Suite

Before physical implementation, comprehensive simulation was performed to verify the logical correctness of the design. This phase included:

\begin{enumerate}
    \item \textbf{Truth Table Verification} -- Validating all operations against expected outputs
    \item \textbf{Timing Analysis} -- Ensuring proper signal propagation and no race conditions
    \item \textbf{Edge Case Testing} -- Identifying potential design flaws with boundary conditions
    \item \textbf{Integration Testing} -- Verifying module interactions
    \item \textbf{Iterative Refinement} -- Making improvements based on simulation results
\end{enumerate}

\section{Hardware Implementation}

After successful verification through circuit simulation, the 4-bit Arithmetic Logic Unit (ALU) was realized as a physical, functional prototype. The implementation utilized a breadboard environment to facilitate easy assembly, modification, and debugging.

\subsection{Components List}

\begin{table}
\captionabove{Hardware Components Used in Implementation}
\centering
\begin{tabular}{lll}
\toprule
\textbf{Sl. No} & \textbf{Component} & \textbf{Specification} \\
\midrule
1 & Logic Gates & 74LS08 (AND), 74LS32 (OR), 74LS86 (XOR) \\
2 & Multiplexers & 74LS151 (8-to-1 MUX) \\
3 & Full Adder & 74LS83 (4-bit binary full adder) \\
4 & Flip-Flops & 74LS74 (D-type flip-flops) \\
5 & Display & LED indicators \\
6 & Power Supply & 5V DC regulated \\
7 & Breadboard & Solderless breadboard \\
8 & Wires & Jumper wires \\
\bottomrule
\end{tabular}
\label{tab:components}
\end{table}

\section{Testing and Troubleshooting}

Verification of circuit operation was performed through systematic testing procedures:

\begin{enumerate}
    \item \textbf{Unit Testing} -- Functional testing of individual modules
    \item \textbf{Integration Testing} -- Testing the complete ALU system
    \item \textbf{Comparison Testing} -- Comparing hardware results with simulation
    \item \textbf{Stress Testing} -- Testing with various input combinations
    \item \textbf{Debugging} -- Identifying and resolving discrepancies
\end{enumerate}

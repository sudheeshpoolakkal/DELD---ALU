\chapter{Abstract}
\begin{tabular}{@{} >{\bfseries}p{0.2\textwidth} p{0.77\textwidth} @{}}
\MakeUppercase{Keywords} &
ALU; Arithmetic Logic Unit; Digital Electronics; Logic Gates; 4-bit; 
Adder; Subtractor; Multiplier; Divider; Binary Operations; Proteus; 
Logic Design; Breadboard Implementation\\
\end{tabular}
\par

\begin{doublespacing}
\textbf{Problem Statement:} Modern computing relies fundamentally on the Arithmetic Logic Unit (ALU) for all data processing. This project addressed the need to practically realize and demonstrate the core operational and architectural principles of a processor's central component, starting from the most basic digital building blocks.

\textbf{Objectives:} The primary objective was to design and implement a comprehensive 4-bit ALU using discrete digital components, specifically basic logic gates and flip-flops. The unit was designed to execute essential arithmetic operations (addition, subtraction, multiplication, and division) and foundational bitwise logical operations (AND, OR, XOR). The project also explicitly aimed to enhance the understanding of digital logic design by showcasing the internal structure of these data manipulation circuits.

\textbf{Methodology:} The design utilized a modular approach. Sub-circuits, such as the 4-bit ripple-carry adder (for addition and subtraction via two's complement) and dedicated multiplication/division logic, were constructed using AND, OR, and XOR gates. Flip-flops were integrated for control and state management. A control unit, built with multiplexers and decoders, governed the selection of the desired operation. Rigorous testing was employed via simulation to verify the functionality and accuracy of all implemented operations against the required truth tables.

\textbf{Key Results:} The project successfully delivered a functioning, modular 4-bit ALU prototype. The implementation validated core principles of digital logic design, demonstrating how simple gates are combined into complex, functional processing units capable of fundamental data manipulation. This work serves as an essential foundation for understanding CPU internal architecture.
\end{doublespacing}
